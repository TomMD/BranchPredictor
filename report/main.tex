\documentclass[11pt]{article}

\usepackage{amsmath,verbatim,multicol}
\usepackage[compact]{titlesec}

\setlength{\parskip}{5pt}
\setlength{\parindent}{0pt}

\title{Alpha 21264 Branch Predictor}
\author{Thomas DuBuisson and Garrett Morris}
\date{March 19, 2012}

\begin{document}
\maketitle

This document describes our simulation of the Alpha 21264 branch predictor.  It summarizes our
interpretation of the branch prediction algorithm, our implementation, and the results on the
provided trace files.

\section{Algorithm}

We implement a tournament branch predictor using both local and global branch histories.  The global
history maps 12 bits of branch history to a two-bit saturating counter, used for prediction.  The
local histories consist of a 2-stage mapping: first, the last 10 bits of the PC are mapped to 10
bits of local branch history.  Each local branch history is then mapped to a three-bit saturating
counter, used for prediction.  Finally, we store a choice prediction table, mapping the global path
history (12 bits) to a two-bit saturating counter, which tracks whether the global or local path
histories are more likely to be accurate for particular paths.  This table is used to choose between
the global and local prediction for each individual branch.

\section{Implementation}

Our implementation stores both 2- and 3-bit saturating counters in 8-bit unsigned integers; we use a
standard set of macros to perform counter operations on the 8-bit values.  We represent the 12-bit
path history using a 16-bit unsigned value.

On the first invocation of the predictor, we initialize all global and local histories.  Depending
on compile-time options, we initialize all histories to either strongly not-taken (0) or
strongly-taken (3 or 7, depending on counter size).  We initialize all choice histories to strongly
local (0), and initialize the path history to either all not taken or all taken.

Our space usage is as follows, where all lines but the total are in bits:

\begin{tabular}{|l|r|rr|rr|}
\hline
\it Table & \it Cells & \multicolumn{2}{|c|}{\it Cell size} & \multicolumn{2}{|c|}{\it Total size} \\
 & & \it needed & \it used & \it needed & \it used \\
\hline
Local history table & 1024 & 10 & 16 & 10240 & 16384 \\
Local predictors & 1024 & 3 & 8 & 3072 & 8192 \\
Global predictors & 4096 & 2 & 8 & 8192 & 32768 \\
Choice predictors & 4096 & 2 & 8 & 8192 & 32768 \\
\hline
\multicolumn{4}{|l|}{\it Total (bytes)} & \it 3712 & \it 11264 \\
\hline
\end{tabular}

\section{Testing Strategy}

The implementation of most of the individual operations was straightforward, and we performed manual
inspection to ensure that they behaved properly.  To test the behavior of the system as a whole, we
compared results obtained by running our predictor against known-good results provided by the course
staff.  These comparisons confirmed that our predictor conformed to the intended behavior.

\section{Results}

\begin{tabular}{|l|rrrr|}
\hline
\it Trace & \it Score & \it Branches & \it CC Branches & \it Predictions \\
\hline
DIST-FP-1 & 3.360 & 2616522 & 2213673 & 2213673 \\
DIST-FP-2 & 1.322 & 1805454 & 1792835 & 1792835 \\
DIST-FP-3 & 0.524 & 1573540 & 1546797 & 1546797 \\
DIST-FP-4 & 0.302 & 921402 & 895842 & 895842 \\
DIST-FP-5 & 2.002 & 2722674 & 2422049 & 2422049 \\
DIST-INT-1 & 7.331 & 4989048 & 4184792 & 4184792 \\
DIST-INT-2 & 11.039 & 3671448 & 2866495 & 2866495 \\
DIST-INT-3 & 12.392 & 4136429 & 3771697 & 3771697 \\
DIST-INT-4 & 2.699 & 2432848 & 2069894 & 2069894 \\
DIST-INT-5 & 0.427 & 3818636 & 3755315 & 3755315 \\
DIST-MM-1 & 8.425 & 2784449 & 2229289 & 2229289 \\
DIST-MM-2 & 11.118 & 4213315 & 3809780 & 3809780 \\
DIST-MM-3 & 2.779 & 4953146 & 3014850 & 3014850 \\
DIST-MM-4 & 2.273 & 5084440 & 4874888 & 4874888 \\
DIST-MM-5 & 6.970 & 3406316 & 2563897 & 2563897 \\
DIST-SERV-1 & 17.567 & 5607148 & 3660616 & 3660616 \\
DIST-SERV-2 & 18.325 & 5443962 & 3537562 & 3537562 \\
DIST-SERV-3 & 9.754 & 5358488 & 3811906 & 3811906 \\
DIST-SERV-4 & 15.978 & 6299843 & 4266796 & 4266796 \\
DIST-SERV-5 & 17.473 & 6411192 & 4291964 & 4291964 \\

\hline
\end{tabular}

\end{document}
